%----Language
%\usepackage[francais,english]{babel}
\usepackage[english]{babel}

%To be able to insert comments, use verbatim.
\usepackage{verbatim}
\usepackage{rotating}
\usepackage{algpseudocode}
\usepackage{booktabs}

\usepackage[usenames,dvipsnames]{xcolor}
\usepackage{xspace}
\usepackage[colorlinks=true,allcolors=blue,breaklinks,draft=false]{hyperref}

\usepackage{caption}

%to get LASY symbols
\usepackage{graphicx}
\usepackage{rotating}
%\usepackage{epsfig}
%\usepackage[T1]{fontenc}
%\usepackage{lmodern}
\usepackage{euscript}
\usepackage{latexsym}
\usepackage{fancyhdr}
\usepackage{tikz}
\usepackage{url}
\usepackage{amssymb}
\usepackage{amsfonts}
\usepackage{amsmath}
%\usepackage{subfigure}
\usepackage{mathpartir}
\usepackage{enumitem}      % adjust spacing in enums

% normalem garantit que \em reste l'italique
\usepackage[normalem]{ulem}


%to support C programs, cprog must be installed: http://www.ctan.org/pkg/cprog
\usepackage{cprog}
\usepackage{listings}

\newcommand*\mycirc[1]{%
      \begin{tikzpicture}[baseline=(C.base)]
        \node[draw,circle,inner sep=1pt](C) {#1};
      \end{tikzpicture}}


      \newcommand{\mc}[3]{\multicolumn{#1}{#2}{#3}}

%%%%%%%%%%%%%%%%%%%%%%%%%%%%%% CARP related includes %%%%%%%%%%%%%%%%%%%%%%%%%%%
% This gives syntax highlighting in the python environment
\renewcommand{\lstlistlistingname}{Code Listings}
\renewcommand{\lstlistingname}{Code Listing}

\definecolor{gray}{gray}{0.5}
\definecolor{key}{rgb}{0,0.5,0}



\def\OPTL{\textrm{$[$}}
\def\OPTR{\textrm{$]$}}

\definecolor{lightbackground}{rgb}{.98,.98,.97}
\definecolor{darkgray}{rgb}{.3,.3,.3}
\definecolor{darkred}{rgb}{.6,0,0}
\definecolor{darkgreen}{rgb}{0,.6,0}
\definecolor{darkblue}{rgb}{0,0,.6}
\usepackage{listings}

\lstdefinelanguage{pencil}{%
  %% Definition du langage
  %% List of keywords
  keywords={[1]},%for,do,while,if,else,break,continue,return
  keywords={[2]struct,union,enum,typedef,volatile,%
    signed,unsigned,sizeof,typeof,inline,noinline,%
    void,char,short,long,int,float,double,boolean,size_t},
  keywords={[3]pragma,pencil,independent,ivdep,reduction,access, private},
  keywords={[4],__attribute__,%
    pencil_access,__pencil_kill,KILL,__pencil_assume,ASSUME,__pencil_assert,%
    __pencil_maybe,MAYBE,__pencil_use,USE,__pencil_def,DEF,MAY_DEF,%
    PENCIL,__pencil_access,ACCESS,CONST,pencil_heap,const,restrict,static, pencil_attributes,
__builtin_assume, __assume, __builtin_unreachable},
  %emph={main,producer,consumer,master,selector,compute,hscan,sync_av},
  %% List of abbreviations
  % literate={<=}{{$\leq$}}1 {>=}{{$\geq$}}1 {!=}{{$\neq$}}1 {*}{{$\times$}}1,
  literate={[OPT[}{{\OPTL}}1 {]OPT]}{{\OPTR}}1,
  %{\#}{{\textbf{\color{darkgreen}\#}}}1,
  %% List of strings
  string=[b]",
  %% List of comment strings
  comment=[l]//,
  morecomment=[s]{/*}{*/},
  %% Special character for LaTeX
  mathescape=true,
  %% Definition du style
  flexiblecolumns=true,
  tabsize=2,
  %% numerotage des lignes
  %firstnumber=1,
  %stepnumber=1,
  %numbers=left,
  % numbersep=-6mm,
  %% titre
  captionpos=b,
  % abovecaptionskip=3mm,
  % belowcaptionskip=3mm,
  %% La boite englobante
  frame=single,
  framerule=0pt,
  aboveskip=1pt,
  belowskip=1pt,
  framesep=1pt,
  %% Les styles
  basicstyle=\scriptsize\ttfamily,
  keywordstyle={[1]\color{darkred}},
  keywordstyle={[2]\color{blue}},
  keywordstyle={[3]\color{darkgreen}\bfseries},
  keywordstyle={[4]\color{darkblue}\bfseries},
  %keywordstyle=\fontseries{bx}\fontfamily{cmss}\fontshape{n}\selectfont,
  % numberstyle=\footnotesize,
  % basicstyle=,
  % keywordstyle=\sbf,
  % numberstyle=,
  emphstyle=\slshape,
  identifierstyle=\color{black},
  commentstyle=\color{darkgray},
  stringstyle=\color{green}
}

%\lstset{language=pencil,backgroundcolor=\color{lightbackground}}
\lstset{language=pencil,backgroundcolor=\color{lightbackground},%
  belowskip=.5em, aboveskip=.5em}

\definecolor{voblacomment}{rgb}{0,0.4,0}
\lstdefinelanguage{vobla}{morekeywords={%
    in,out,as,is,im,sum,let,range,for, forall,while,if,else,len,this,base,%
    export,function,interface,parameter,implements,storage,layout,view,%
    Reversed,Transpose,Normal,Diag,AntiDiag,Conjugated,Row,Column,Complex,%
    Re,Im,Transpose,Conjugate,Diagonal,AntiDiagonal,Sparse,%
  Double,Real,Float,Index,Value,import},%
  comment=[l]//,
  %% Definition du style
  flexiblecolumns=true,
  tabsize=2,
  %% titre
  captionpos=b,
  %% La boite englobante
  frame=single,
  framerule=0pt,
  aboveskip=10pt,
  belowskip=10pt,
  framesep=1pt,
  %% Les styles
  basicstyle=\ttfamily\small,
  keywordstyle=\color{blue},
  %keywordstyle=\fontseries{bx}\fontfamily{cmss}\fontshape{n}\selectfont,
  % numberstyle=\footnotesize,
  % basicstyle=,
  % keywordstyle=\sbf,
  % numberstyle=,
  commentstyle=\color{voblacomment},
}

% a few macros to avoid typo
\newcommand\carp{\textsc{Carp}}


\let\ctextfont=\ttfamily
\lstset{basicstyle=\small\ttfamily}

\newcommand{\LB}{\textrm{\L}}

\newcommand\isl{\texttt{isl}\xspace}
\newcommand\pet{\texttt{pet}\xspace}
\newcommand\PPCG{\texttt{PPCG}\xspace}
\newcommand\pencil{\textsc{Pencil}\xspace}

\newcommand{\lstnumberautorefname}{Line}

% Operational semantics macros
\newcommand{\lplaineval}{\Downarrow}
\newcommand{\leval}[2]{{#1}\lplaineval{#2}}
\newcommand{\lfor}[5]{{#1} \  \texttt{for} ( #2; \, #3; \, #4 ) \, #5}
\newcommand{\lufor}[5]{{#1} \  \underline{\mathtt{for}} ( #3; \, #4 ) \, #5}
\newcommand{\lassume}[1]{\texttt{\_\_pencil\_assume} ( #1 )}
\newcommand{\lseq}[2]{#1 \ #2}
\newcommand{\ltrue}{\mathsf{true}}
\newcommand{\lfalse}{\mathsf{false}}
\newcommand{\lstate}{\sigma}
\newcommand{\lreadset}{\mathcal{R}}
\newcommand{\lwriteset}{\mathcal{W}}
\newcommand{\lintuple}[2]{\langle #1, #2 \rangle}
\newcommand{\louttuple}[3]{\langle #1, #2, #3 \rangle}
\newcommand{\leouttuple}[4]{\langle #1, #2, #3, #4 \rangle}
\newcommand{\lclearscope}[2]{\kappa(#1)}
\newcommand{\lcheckrwsetsplain}{\mathrm{check\_rw}}
\newcommand{\lcheckrwsets}[5]{\mathrm{check\_rw}_{#1}(#2, #3, #4, #5)}
