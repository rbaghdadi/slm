% llncs : version 2.4 for LaTeX2e as of 16. April 2010
\documentclass{llncs}
%

\usepackage{amsmath}
\usepackage{nomencl}
%----Language
%\usepackage[francais,english]{babel}
\usepackage[english]{babel}

%To be able to insert comments, use verbatim.
\usepackage{verbatim}
\usepackage{rotating}
\usepackage{algpseudocode}
\usepackage{booktabs}

\usepackage[usenames,dvipsnames]{xcolor}
\usepackage{xspace}
\usepackage[colorlinks=true,allcolors=blue,breaklinks,draft=false]{hyperref}

\usepackage{caption}

%to get LASY symbols
\usepackage{graphicx}
\usepackage{rotating}
%\usepackage{epsfig}
%\usepackage[T1]{fontenc}
%\usepackage{lmodern}
\usepackage{euscript}
\usepackage{latexsym}
\usepackage{fancyhdr}
\usepackage{tikz}
\usepackage{url}
\usepackage{amssymb}
\usepackage{amsfonts}
\usepackage{amsmath}
%\usepackage{subfigure}
\usepackage{mathpartir}
\usepackage{enumitem}      % adjust spacing in enums

% normalem garantit que \em reste l'italique
\usepackage[normalem]{ulem}


%to support C programs, cprog must be installed: http://www.ctan.org/pkg/cprog
\usepackage{cprog}
\usepackage{listings}

\newcommand*\mycirc[1]{%
      \begin{tikzpicture}[baseline=(C.base)]
        \node[draw,circle,inner sep=1pt](C) {#1};
      \end{tikzpicture}}


      \newcommand{\mc}[3]{\multicolumn{#1}{#2}{#3}}

%%%%%%%%%%%%%%%%%%%%%%%%%%%%%% CARP related includes %%%%%%%%%%%%%%%%%%%%%%%%%%%
% This gives syntax highlighting in the python environment
\renewcommand{\lstlistlistingname}{Code Listings}
\renewcommand{\lstlistingname}{Code Listing}

\definecolor{gray}{gray}{0.5}
\definecolor{key}{rgb}{0,0.5,0}



\def\OPTL{\textrm{$[$}}
\def\OPTR{\textrm{$]$}}

\definecolor{lightbackground}{rgb}{.98,.98,.97}
\definecolor{darkgray}{rgb}{.3,.3,.3}
\definecolor{darkred}{rgb}{.6,0,0}
\definecolor{darkgreen}{rgb}{0,.6,0}
\definecolor{darkblue}{rgb}{0,0,.6}
\usepackage{listings}

\lstdefinelanguage{pencil}{%
  %% Definition du langage
  %% List of keywords
  keywords={[1]},%for,do,while,if,else,break,continue,return
  keywords={[2]struct,union,enum,typedef,volatile,%
    signed,unsigned,sizeof,typeof,inline,noinline,%
    void,char,short,long,int,float,double,boolean,size_t},
  keywords={[3]pragma,pencil,independent,ivdep,reduction,access, private},
  keywords={[4],__attribute__,%
    pencil_access,__pencil_kill,KILL,__pencil_assume,ASSUME,__pencil_assert,%
    __pencil_maybe,MAYBE,__pencil_use,USE,__pencil_def,DEF,MAY_DEF,%
    PENCIL,__pencil_access,ACCESS,CONST,pencil_heap,const,restrict,static, pencil_attributes,
__builtin_assume, __assume, __builtin_unreachable},
  %emph={main,producer,consumer,master,selector,compute,hscan,sync_av},
  %% List of abbreviations
  % literate={<=}{{$\leq$}}1 {>=}{{$\geq$}}1 {!=}{{$\neq$}}1 {*}{{$\times$}}1,
  literate={[OPT[}{{\OPTL}}1 {]OPT]}{{\OPTR}}1,
  %{\#}{{\textbf{\color{darkgreen}\#}}}1,
  %% List of strings
  string=[b]",
  %% List of comment strings
  comment=[l]//,
  morecomment=[s]{/*}{*/},
  %% Special character for LaTeX
  mathescape=true,
  %% Definition du style
  flexiblecolumns=true,
  tabsize=2,
  %% numerotage des lignes
  %firstnumber=1,
  %stepnumber=1,
  %numbers=left,
  % numbersep=-6mm,
  %% titre
  captionpos=b,
  % abovecaptionskip=3mm,
  % belowcaptionskip=3mm,
  %% La boite englobante
  frame=single,
  framerule=0pt,
  aboveskip=1pt,
  belowskip=1pt,
  framesep=1pt,
  %% Les styles
  basicstyle=\scriptsize\ttfamily,
  keywordstyle={[1]\color{darkred}},
  keywordstyle={[2]\color{blue}},
  keywordstyle={[3]\color{darkgreen}\bfseries},
  keywordstyle={[4]\color{darkblue}\bfseries},
  %keywordstyle=\fontseries{bx}\fontfamily{cmss}\fontshape{n}\selectfont,
  % numberstyle=\footnotesize,
  % basicstyle=,
  % keywordstyle=\sbf,
  % numberstyle=,
  emphstyle=\slshape,
  identifierstyle=\color{black},
  commentstyle=\color{darkgray},
  stringstyle=\color{green}
}

%\lstset{language=pencil,backgroundcolor=\color{lightbackground}}
\lstset{language=pencil,backgroundcolor=\color{lightbackground},%
  belowskip=.5em, aboveskip=.5em}

\definecolor{voblacomment}{rgb}{0,0.4,0}
\lstdefinelanguage{vobla}{morekeywords={%
    in,out,as,is,im,sum,let,range,for, forall,while,if,else,len,this,base,%
    export,function,interface,parameter,implements,storage,layout,view,%
    Reversed,Transpose,Normal,Diag,AntiDiag,Conjugated,Row,Column,Complex,%
    Re,Im,Transpose,Conjugate,Diagonal,AntiDiagonal,Sparse,%
  Double,Real,Float,Index,Value,import},%
  comment=[l]//,
  %% Definition du style
  flexiblecolumns=true,
  tabsize=2,
  %% titre
  captionpos=b,
  %% La boite englobante
  frame=single,
  framerule=0pt,
  aboveskip=10pt,
  belowskip=10pt,
  framesep=1pt,
  %% Les styles
  basicstyle=\ttfamily\small,
  keywordstyle=\color{blue},
  %keywordstyle=\fontseries{bx}\fontfamily{cmss}\fontshape{n}\selectfont,
  % numberstyle=\footnotesize,
  % basicstyle=,
  % keywordstyle=\sbf,
  % numberstyle=,
  commentstyle=\color{voblacomment},
}

% a few macros to avoid typo
\newcommand\carp{\textsc{Carp}}


\let\ctextfont=\ttfamily
\lstset{basicstyle=\small\ttfamily}

\newcommand{\LB}{\textrm{\L}}

\newcommand\isl{\texttt{isl}\xspace}
\newcommand\pet{\texttt{pet}\xspace}
\newcommand\PPCG{\texttt{PPCG}\xspace}
\newcommand\pencil{\textsc{Pencil}\xspace}

\newcommand{\lstnumberautorefname}{Line}

% Operational semantics macros
\newcommand{\lplaineval}{\Downarrow}
\newcommand{\leval}[2]{{#1}\lplaineval{#2}}
\newcommand{\lfor}[5]{{#1} \  \texttt{for} ( #2; \, #3; \, #4 ) \, #5}
\newcommand{\lufor}[5]{{#1} \  \underline{\mathtt{for}} ( #3; \, #4 ) \, #5}
\newcommand{\lassume}[1]{\texttt{\_\_pencil\_assume} ( #1 )}
\newcommand{\lseq}[2]{#1 \ #2}
\newcommand{\ltrue}{\mathsf{true}}
\newcommand{\lfalse}{\mathsf{false}}
\newcommand{\lstate}{\sigma}
\newcommand{\lreadset}{\mathcal{R}}
\newcommand{\lwriteset}{\mathcal{W}}
\newcommand{\lintuple}[2]{\langle #1, #2 \rangle}
\newcommand{\louttuple}[3]{\langle #1, #2, #3 \rangle}
\newcommand{\leouttuple}[4]{\langle #1, #2, #3, #4 \rangle}
\newcommand{\lclearscope}[2]{\kappa(#1)}
\newcommand{\lcheckrwsetsplain}{\mathrm{check\_rw}}
\newcommand{\lcheckrwsets}[5]{\mathrm{check\_rw}_{#1}(#2, #3, #4, #5)}


%
\begin{document}
%
\frontmatter          % for the preliminaries
\pagestyle{headings}  % switches on printing of running heads
\mainmatter           % start of the contributions
%
\title{Automatic Parallelization and Optimization of a
	SLAM Application Written in the PENCIL Language}

%
\author{}

\institute{INRIA and \'Ecole Normale Sup\'erieure \\
\email{\textit{first}.\textit{last}@inria.fr}}

\maketitle{}

\begin{abstract}
  Low-level GPU programming with CUDA and OpenCL is difficult, error
  prone, and not performance-portable. Automatic parallelization
  and code optimization techniques have been proposed
  to hide this complexity and regain some performance portability.
  \pencil{} is a higher level, platform-neutral compute intermediate
  language, designed to improve the ability of compilers to better
  parallelize and optimize code when targeting accelerators.

  In this paper, we report our experience in using automatic
  parallelization and optimization techniques of a challenging,
  real world application written in \pencil{}:
  the KinectFusion application, a large computer
  vision and \textit{simultaneous localization and mapping} (SLAM)
  application.
  We use a polyhedral compiler to map the \pencil{} code into
  an OpenCL code.
  We demonstrate how \pencil{} helps in enhancing the ability of
  that polyhedral compiler to automatically parallelize and optimize
  the SLAM application and we show that we obtain acceptable performance
  compared to the relatively low programming effort and compared to
  the high portability benefit.
\end{abstract}

\section{Introduction}

Special-purpose \emph{accelerators} such as GPUs can significantly
outperform general-purpose processors in terms of raw performance and
energy efficiency.  Software for such accelerators is currently
written using low-level APIs and target-specific languages, such as
OpenCL and CUDA, which increases the cost of its development and
maintenance.

Computer vision is a typical domain that relies extensively on the
use of accelerators and where writing low level code to target
different architectures is becoming more and more difficult, error
prone and featuring non-portable optimizations.
On the other hand, writing high level code and lowering that code
through an optimizing compiler naturally arose as a viable alternative.

\pencil{}~\cite{pencil}, a platform-neutral compute intermediate
language is an example of a language that can be used to target
accelerators.
Although the target platforms are highly parallel, \pencil{} has
sequential semantics.  In addition, it is equiped with language
constructs and specific properties particularly well
suited for static analysis of array computations.
To compile \pencil{} programs to OpenCL, the polyhedral framework is used,
a program representation and a set of analyses, transformations and code
generation algorithms that enable a compiler to reason about advanced
loop nest transformations addressing most of the parallelism and
locality-enhancing challenges.

To understand the tradeoffs behind the design of such an a
language, we study 


In this paper, we evaluate \pencil{} on a benchmark from the computer
vision and the \textit{simultaneous localization and mapping} (SLAM)
domains.
This benchmark is particularly interesting as it represents a realistic
application that has code patterns that are challenging for a classical
polyhedral compiler, such as irregular, data-dependent control flow
and array accesses.
We compare the performance of the automatically generated OpenCL codes
with the reference OpenCL manual implementation of SLAMBench.

Although our study focuses on examples from the domain of computer
vision, the proposed language and the \pencil{} compilation flow
is generally applicable to a wide set of array-based computational
problems.
Separate evaluations on the fields of image processing, linear algebra,
signal processing and on the SHOC and Rodinia benchmarks are being
conducted separately.


Overview of PENCIL



\section{Overview of the SLAMBench Computer Vision Benchmark}

To explore the applicability of \pencil to a complex,
real-world application, a \pencil port for the SLAMBench
Benchmark was implemented~\cite{SLAMBench2015}.
SLAMBench provides C++, OpenMP, OpenCL, and CUDA implementations
of KinectFusion~\cite{KinectFusion2011}, a dense computer vision
and \textit{simultaneous localization and mapping} (SLAM)
application
SLAMBench was developed as part of the PAMELA project, that
intends to take a holistic approach to the development
of many-core architectures, compilers, and runtime systems,
while focusing on the computer vision application
domain~\cite{PAMELAWeb2012}.

SLAMBench is composed of 12 key kernels (listed in
Table~\ref{SLAMBench_Features}). 
%: \textit{mm2meters, bilateralFilter, integrate, depth2vertex,
% vertex2normal, halfSampleRobustImage, renderDepth, renderTrack,
% renderVolume, raycast, track} and \textit{reduce}. 
Ten of the twelve can be classified as being irregular:
All kernels except \textit{depth2vertex}, \textit{vertex2Normal},
and \textit{mm2meters} exhibit non static-affine conditions and/or
loop-bounds, while \textit{raycast}, \textit{renderVolume},
\textit{vertex2Normal}, \textit{integrate}, and
\textit{bilateralFilter} also exhibit non static-affine array reads.


Why is it interesting.


Show code snapshots.


\section{Overview of \pencil{}}



\section{Polyhedral Compilation of PENCIL Code}



\section{Evaluation}
\label{slambench}

The benchmark reports results for 5 distinct phases, each of which
is composed of one or more of the above kernels.
For each phase, we compare performance against the pre-existing OpenCL
implementation that has been hand-optimized for the ARM Arndale board
with the Mali-T604 GPU. The phases and their constituent kernels are
as follows:

\begin{itemize}
\item \textbf{\textit{preprocessing}:} \textit{mm2meters, bilateralFilter};
\item \textbf{\textit{integration}:} \textit{integrate};
\item \textbf{\textit{raycasting}:} \textit{raycast};
\item \textbf{\textit{rendering}:} \textit{renderDepth, renderTrack, renderVolume};
\item \textbf{\textit{tracking}:} \textit{depth2vertex, vertex2normal,
      halfSampleRobustImage, track, reduce}.
\end{itemize}


Methodology


Platforms


Table~\ref{SLAMBench_Speedups} presents the results for our optimized
\pencil implementation, compiled using the best \PPCG tile and
work-group size parameters that could be identified with reasonable
effort.
We present both the wall-clock times as well as only the total kernel
execution times in seconds for both the hand-optimized OpenCL and
\pencil implementations.
The \pencil implementation's wall-clock performance is only
$0.31\times$ that of the hand-optimized OpenCL.

The key reason for this is the current lack of support for whole-program
optimization, particulary with respect to memory allocation across host
and device memory.
The hand-optimized OpenCL judiciously allocates memory between host and
device memory, minimizing unnecessary copies between kernel invocations.
Unfortunately, OpenCL generated from \pencil is compiled as a separate
library that is invoked from within the C++ SLAMBench code.
Therefore, \PPCG is only able to optimize and transform memory access
patterns within each kernel, independently of the other kernels. 
Thus each runtime kernel invocation requires copying of the relevant
data-structures from host to device memory, and then back. 
As each of the kernels is called either 882 or 2646 times (for the
default dataset), the cumulative effect of these repeated copies
significantly reduces wall-clock performance relative to hand-tuned OpenCL.
Nevertheless, despite the $3\times$ overall slow-down, \pencil SLAMBench
wall-clock performance is still roughly $2.4\times$ better than the
sequential C++ implementation.

\begin{table}[htbp]
\begin{center}
\scalebox{0.85}{
\begin{tabular}{|l|r|r|r|r|r|r|}
\hline
& \multicolumn{2}{c|}{\textbf{OpenCL}} & \multicolumn{2}{c|}{\textbf{PENCIL}} & \multicolumn{2}{c|}{\textbf{SpeedUp}} \\ \hline
\textbf{Phases} & \textbf{clock} & \textbf{kernel} & \textbf{clock} & \textbf{kernel} & \textbf{clock} & \textbf{kernel} \\ \hline
preprocessing & 16.2 & 10.5 & 23.8 & 11.4 & $0.68\times$ & $0.92\times$ \\ %\hline
integration & 69.3 & 67.7 & 403.7 & 93.7 & $0.17\times$ & $0.72\times$ \\ %\hline
raycasting & 65.8 & 64.8 & 97.3 & 56.7 & $0.68\times$ & $1.14\times$ \\ %\hline
rendering & 23.3 & 17.9 & 44.8 & 15.7 & $0.52\times$ & $1.14\times$ \\ %\hline
tracking & 98.9 & 50.2 & 355.1 & 81.5 & $0.28\times$ & $0.62\times$ \\ %\hline
\textbf{Total} & \textbf{279.2} & \textbf{200.5} & \textbf{906.1} & \textbf{259.1} & \textbf{0.31$\times$} & \textbf{0.77$\times$} \\ \hline
\end{tabular} }
\end{center}
\vskip-0.1cm
\caption{Performance comparison of the PENCIL implementation of
         SLAMBench vs. hand-optimized OpenCL, on ARM Mali-T604.}
\label{SLAMBench_Speedups}
\vskip-0.1cm
\end{table}

Furthermore, though the \pencil implementation exhibits only a third
of the wall-clock performance of hand-optimized OpenCL, the kernel-only
execution times are only 23\% slower overall. The \textit{raycasting} and
\textit{rendering} phases even exhibit a 14\% kernel performance improvement
over hand-optimized OpenCL, while the \textit{preprocessing} phase is only
about 8\% slower.
The reason that kernel performance is sometimes better than hand-optimized
OpenCL is that the simplicity and higher-level of abstraction provided by
\pencil make it much easier to more thoroughly explore the optimization
search-space, simply by changing the \PPCG compiler parameters.
This is in contrast to the more involved and careful changes that must
be manually made to OpenCL host and kernel code for each set of tile
and work-group size parameters, or for any of the loop optimizations
and transforms that must be manually implemented without the assistance
of a polyhedral compiler.

Note that unlike the other benchmarks discussed above, the SLAMBench
\pencil compilation flow did not make use of auto-tuning, so the current
best \PPCG parameters were discovered through manual trial-and-error.
With further work, we hope to make use of auto-tuning to further improve
upon the reported kernel performance figures. To improve to wall-clock
performance and improve memory allocation across the system, further
enhancements to both the \pencil language and the \PPCG compiler must
be considered.

Table~\ref{SLAMBench_Features} lists the \pencil features that affect
the implementation and/or performance of each of the ported kernels.
All of the kernels that have non static-affine code require \PPCG's
support for irregular code compilation.
In addition, most such kernels also make use of summary
functions -- innermost loop code was extracted into library functions
that were called from within the \pencil code.
This was needed to incorporate non-polyhedral code optimizations like
vector data-types, that were necessary to improve performance relative
to the hand-optmized OpenCL.

%track: NSCC
%raycast: NSCC, NSRD
%renderVolume: NSCC, NSRD
%renderTrack: NSCC
%renderDepth: NSCC
%halfSampleRobustImage: NSCC
%vertex2Normal: NSRD
%depth2Vertex:
%mm2meters:
%integrateKernel: NSCC, NSRD
%bilateralFilter: NSCC, NSRD
%reduce: NSCC

\begin{table}[htbp]
\centering
\begin{center}
\scalebox{0.75} {
\begin{tabular}{|l|l|l|l|}
\hline                       
Kernel & no support for            & no parameter & no  \\
       & \textit{irregular} code   & tuning &  \textit{assumes} \\
%       & or summary functions      & &  \\       
\hline
\textit{mm2meters}             &     	-	           & $15\% \downarrow$   &  $28\% \downarrow$ \\
\textit{bilateralFilter}       & 	no code	generation &     -               &  $14\% \downarrow$ \\	
\textit{integrate}             & 	no code	generation &	 $51\% \downarrow$   &   -                \\                       
\textit{depth2vertex}          &     	-	           &     -                &  $79\% \downarrow$ \\
\textit{vertex2normal}         & 	no code generation & $58\% \downarrow$    &   -             \\
\textit{halfSampleRobustImage} & 	no code	generation & $14\% \downarrow$   &  $6\% \downarrow$ \\
\textit{renderDepth}           & 	no code	generation &     -               &  $23\% \downarrow$ \\
\textit{renderTrack}           & 	no code	generation &     -               &  $5\% \downarrow$ \\
\textit{renderVolume}          & 	no code	generation & $52\% \downarrow$   &   -	 \\
\textit{raycast}               & 	no code	generation & $51\% \downarrow$   &   -             \\
\textit{track}                 & 	no code	generation & $27\% \downarrow$   &  $4\% \downarrow$ \\
\textit{reduce}                & 	no code	generation & $24\% \downarrow$   & 	-	 \\
\hline
\end{tabular} }
\end{center}
\vskip-0.2cm
\caption{\label{SLAMBench_Features}Effect of not Using \pencil Features on the Speedups}
\vskip-0.5cm
\end{table}  

The \pencil based SLAMBench implementation performance was also
highly sensitive to \PPCG's compiler flags, in particular the
work-group and tile size parameters for each kernel.
The third column of Table~\ref{SLAMBench_Features} presents the
performance impact of compiling each kernel with default
parameters only, relative to execution with the most optimized
set of parameters.
Many of the kernels exhibit performance degradation of $15-60\%$.
Also removing the \lstinline!__pencil_assume! statements further
degrades the performance of some of the kernels by $4-80\%$.

The results in this section demonstrate the potential advantages
of the \pencil language and polyhedral compilation for rapidly
implementing and tuning code for execution on GPUs or other
accelerators, even for a complex, real-world application like
KinectFusion.
We also identify a key remaining issue that need to be addressed
-- optimizing memory allocation and communication across host
and accelerator -- for this to become the preferred approach to
rapid and efficient accelerator programming.

% This is what we do here: (a) a short description of SLAMBench:
% what it is, why it is relevant etc. (PAMELA justification here),
% followed by (b) Characteristics of SLAMBench: it contains irregular
% code, is a full application (i.e. pencil code must be incorporated
% with irregular, non-compute CPP benchmark code), and (c) the
% constituents of SLAMBench - the various kernels, divided into phases.
% some insight into the kernel code itself (esp the key ones: integrate,
% track, reduce).

% For SLAMbench, several features were essential for either getting
% it to work at all or having it perform correctly.
% Summary functions used extensively.
% These were sometimes necessary to overcome some fragility in
% our toolchain, but also proved useful in enabling target-specific,
% non-polyhedral, OpenCL code optimizations like vectorization.
% Also used: assumes. independent directive was not needed.
% kills not used, but could've helped.


\section{Related Work}


\section{Conclusion}


\bibliographystyle{abbrv}
\bibliography{bibliographies/pencilbib,bibliographies/SLAMBenchCitations}

\end{document}
